\documentclass[a4paper,12pt]{article}

%%% Работа с русским языком
\usepackage{cmap}					% поиск в PDF
\usepackage{mathtext} 				% русские буквы в фомулах
\usepackage[T2A]{fontenc}			% кодировка
\usepackage[utf8]{inputenc}			% кодировка исходного текста
\usepackage[english,russian]{babel}	% локализация и переносы
\usepackage[left=25mm, top=15mm, right=20mm, bottom=15mm, nohead, footskip=10mm]{geometry}
%%% Дополнительная работа с математикой
\usepackage{amsfonts,amssymb,amsthm,mathtools} % AMS
\usepackage{amsmath}
\usepackage{icomma} % "Умная" запятая: $0,2$ --- число, $0, 2$ --- перечисление

%% Номера формул
%\mathtoolsset{showonlyrefs=true} % Показывать номера только у тех формул, на которые есть \eqref{} в тексте.

%% Шрифты
\usepackage{euscript}	 % Шрифт Евклид
\usepackage{mathrsfs} % Красивый матшрифт

%% Свои команды
\DeclareMathOperator{\sgn}{\mathop{sgn}}

%% Перенос знаков в формулах (по Львовскому)
\newcommand*{\hm}[1]{#1\nobreak\discretionary{}
{\hbox{$\mathsurround=0pt #1$}}{}}

%%% Работа с картинками
\usepackage{graphicx}  % Для вставки рисунков
\graphicspath{{images/}{images2/}}  % папки с картинками
\setlength\fboxsep{3pt} % Отступ рамки \fbox{} от рисунка
\setlength\fboxrule{1pt} % Толщина линий рамки \fbox{}
\usepackage{wrapfig} % Обтекание рисунков и таблиц текстом

%%% Работа с таблицами
\usepackage{array,tabularx,tabulary,booktabs} % Дополнительная работа с таблицами
\usepackage{longtable}  % Длинные таблицы
\usepackage{multirow} % Слияние строк в таблице
%%% Ссылки внутри документа
\usepackage[unicode, pdftex]{hyperref}
%%% Заголовок
\usepackage{graphicx}
\usepackage{wrapfig}
\usepackage{listings}
\usepackage{color}

\definecolor{dkgreen}{rgb}{0,0.6,0}
\definecolor{gray}{rgb}{0.5,0.5,0.5}
\definecolor{mauve}{rgb}{0.58,0,0.82}

\lstset{frame=tb,
  language=Java,
  aboveskip=3mm,
  belowskip=3mm,
  showstringspaces=false,
  columns=flexible,
  basicstyle={\small\ttfamily},
  numbers=none,
  numberstyle=\tiny\color{gray},
  keywordstyle=\color{blue},
  commentstyle=\color{dkgreen},
  stringstyle=\color{mauve},
  breaklines=true,
  breakatwhitespace=true,
  tabsize=3
}


\begin{document} 
% Set default path for images
\graphicspath{ {./imgs/} }
\title{Обзор возможностей матричного подхода к некоторым задачам о перестановках}
\author{https://github.com/enlacroix}
\maketitle
\begin{abstract}
Часть 1. Разбор достаточно громоздкого авторского метода, который позволяет 'извлечь квадратный корень' из перестановки и найти коммутирующие с ней. Часть 2. Обзор некоторых продвинутых понятий, связанных с перестановками - сопряжённость и различные формулы подсчёта. 
\end{abstract}
\begin{center}
    \textsc{1. Задача о нахождении квадратного корня из перестановки}
\end{center}
Уточним, что же имеется в виду под извлечением квадратного корня. Речь идет о нахождении \textbf{всех} таких перестановок $x$ для заданного $s$, что $x \circ x = x^2 = s.$ 
Сопоставим перестановке $p \in S_n$ матрицу размера $n \times n$ $A_p$ по следующему правилу: в $i$-строке матрицы все элементы нули, кроме единицы на месте $a_i$, где $a_i$ - элемент перестановки.
\[p = 3, 1, 4, 2 \Rightarrow A_p = \begin{pmatrix}
0 & 0 & 1 & 0 \\
1 & 0 & 0 & 0 \\
0 & 0 & 0 & 1 \\
0 & 1 & 0 & 0 
\end{pmatrix} \]
Свойства: 
\begin{itemize}
    \item Сохранение умножения: Если $p = qr$, то $A_p = A_q \cdot A_r$.
    \item Перестановка обратная к $p$ имеет матрицу обратную к $A_p$.
\end{itemize}
\textit{Примечание.} Перестановочная матрица ортогональна, следовательно обратная к ней $A^{-1} = A^T.$ Перестановка, обратная к данной, может быть получена транспонированием матрицы заданной. \\
Переведем нашу задачу на матричный язык. Пусть перестановке $x$ соответствует матрица $A$, а $s$ -- $B$. Тогда имеем
\[ A^2 = B \Rightarrow A = B \cdot A^T\]
Итак, получили матричное уравнение. Суть его в том, что элементы, стоящие на одинаковых позициях, будут одинаковыми в левой и правой матрице. Поэтому попробуем понять, что происходит с элементом $a_{ij}$ матрицы $A$. Шаг первый -- транспонирование. Тогда $(i, j) \longrightarrow (j, i).$ Шаг второй - умножение слева на матрицу перестановки. $k$-ая строка, у которой на $n$-м месте стоит единица, задает перестановку $k \Rightarrow n$. Строка умножается последовательно на каждый столбец $A^T$, поэтому $k$-ая строка новой матрицы будет формироваться из элементов, которые имели $n$-ый номер в столбце матрицы $A^T$. Иными словами, осуществляется перестановка строк в $A^T$. Таким образом, первый индекс остается неизменным, а второй переходит в другое число, согласно перестановке, задаваемой $B$: $(j, i) \longrightarrow (j, \varphi(i))$. Рассмотрим на конкретном простом примере. Пусть B задает перестановку $\varphi: \{231\}$ \\
\[ B \cdot A^T = \begin{pmatrix}
0 & 1 & 0 \\
0 & 0 & 1 \\
1 & 0 & 0 
\end{pmatrix} \cdot \begin{pmatrix}
a_{11} & a_{21} & a_{31} \\
a_{12} & a_{22} & a_{32} \\
a_{13} & a_{23} & a_{33} 
\end{pmatrix} = \begin{pmatrix}
a_{12} & a_{22} & a_{32} \\
a_{13} & a_{23} & a_{33} \\
a_{11} & a_{21} & a_{31} 
\end{pmatrix}\]
Мы пришли к выводу о том, что элемент, стоящий на $(i, j)$ позиции переместится на $(j, \varphi(i))$-юю. Аналогичные рассуждения можно провести и для новых координат. Можно заметить, что элементы, чьи координаты будут принадлежать такому циклу, будут равными. Снова обратимся к нашему примеру:
\[A = \begin{pmatrix}
a_{11} & a_{12} & a_{13} \\
a_{21} & a_{22} & a_{23} \\
a_{31} & a_{32} & a_{33} 
\end{pmatrix} = B \cdot A^T = \begin{pmatrix}
a_{12} & a_{22} & a_{32} \\
a_{13} & a_{23} & a_{33} \\
a_{11} & a_{21} & a_{31} 
\end{pmatrix}\]
Обратите внимание, что, например, раскручивая цепочку равенств $a_{13} = a_{32} = a_{21}$, мы больше ничего не сможем узнать. Мы понимаем из равенства матриц, что эти элементы равны. Нетрудно теперь проверить, что их координаты подчиняются сформулированному ранее правилу перехода:
\[(1, 3) \longrightarrow (3, \varphi(1)) \longrightarrow (2, \varphi(3)) \longrightarrow (1, \varphi(2)) = (1, 3)\]
Можно написать код, который автоматизирует эту простую процедуру. Что программа выведет для \{231\}:
\begin{lstlisting}
[[(1, 1), (1, 2), (2, 2), (2, 3), (3, 3), (3, 1)], [(1, 3), (3, 2), (2, 1)]]
[6, 3]
\end{lstlisting}
Мы получаем два непересекающихся между собой равенства. Для того чтобы разрешить матричное уравнение, мы должны определить какие элементы будут равняться 1. Понятно, что нам не подойдет цикл из 6 координат, поэтому единственно возможный вариант это $a_{13} = a_{32} = a_{21} = 1.$ Важный момент: мы должны проверить, что все индексы попарно различны, так как в конечном итоге эти элементы должны образовать биективную матрицу перестановки. Далее будут встречаться примеры, когда цикл имеет подходящую длину, но нам не подходит. \\
Сортируем элементы по первому индексу: $a_{13}, a_{21}, a_{32}.$ Тогда мы видим, что в первой строке на месте 3 стоит единица, поэтому 1 переходит в 3 и так далее. Выписываем вторую координату в получившемся порядке: \{312\}. Проверяем: $\{312\} \cdot \{312\} = \{231\}$. \\
Рассмотрим пример посложнее. \\
\textbf{Задача 1.} Найти все корни из перестановки \{561234\} $\in S_6$. \\
Необходимые переходы находятся достаточно быстро и вручную. 
\begin{lstlisting}
[[(1, 1), (1, 5), (5, 5), (5, 3), (3, 3), (3, 1)], 
[(1, 2), (2, 5), (5, 6), (6, 3), (3, 4), (4, 1)],
[(1, 3), (3, 5), (5, 1)], [(1, 4), (4, 5), (5, 2), (2, 3), (3, 6), (6, 1)],
[(1, 6), (6, 5), (5, 4), (4, 3), (3, 2), (2, 1)],
[(2, 2), (2, 6), (6, 6), (6, 4), (4, 4), (4, 2)], [(2, 4), (4, 6), (6, 2)]]
[6, 6, 3, 6, 6, 6, 3]
\end{lstlisting}
Обратите внимание на то, что 6-циклы в первой и предпоследней строке нам не подходят, ввиду того, что индексы элементов внутри цикла совпадают. Поэтому всего возможных комбинаций четыре: 6, 6, 6 и 3 + 3. 
Сортируя по первому индексу, и выписывая вторые в нужном порядке, мы сразу получаем искомые переставновки:
\[\{254163\}, \{4365213\}, \{6123453\}, \{3456123\}\]
\begin{center}
    \textsc{2. Задача о нахождении коммутирующих перестановок}
\end{center}
Для заданной перестановки $s$ (матрица $B$) найти все $x$ ($A$), что
\[ x \circ s = s \circ x \Rightarrow A \cdot B = B \cdot A \Rightarrow A = B \cdot A \cdot B^T\]
Наши рассуждения будут аналогичны -- необходимо понять, какие индексы будет иметь элемент $a_{ij}$ в правой части уравнения. Умножение на матрицу $B$ слева осуществляет перестановку строк, а умножение на $B^T$ справа -- перестановку столбцов. Итоговое преобразование запишется, как $(i, j) \longrightarrow (\varphi(i), \varphi(j))$. \\
\textbf{Задача 2.} Найти все перестановки, коммутирующие с $s = \{24513\} = (124)(35)$
\begin{lstlisting}
[[(1, 1), (2, 2), (4, 4)], [(1, 2), (2, 4), (4, 1)], [(1, 4), (2, 1), (4, 2)]
[(1, 3), (2, 5), (4, 3), (1, 5), (2, 3), (4, 5)],
[(3, 1), (5, 2), (3, 4), (5, 1), (3, 2), (5, 4)], 
[(3, 3), (5, 5)], [(3, 5), (5, 3)]]
[3, 3, 6, 3, 6, 2, 2]
\end{lstlisting}
Понятно, что нам подходят только сочетания вида 3 + 2. Имеем $3 \times 2 = 6$ перестановок, коммутирующих с $s$. Упорядочивая по первому индексу, аккуратно выписываем вторые индексы в нужном порядке, они и нужную образуют перестановку. 
\[\{12345\} = s^6, \{12543\} = s^3, \{24315\} = s^4, \{24513\} = s^1, \{41325\} = s^2, \{41523\} = s^5\]
Отметим, что все коммутирующие перестановки исчерпываются перечислением степеней. Действительно, перестановка $s^k$ всегда будет коммутировать с $s$, но множество коммутирующих может состоять и из других представителей. Рассмотрим иллюстрирующий пример крайней похожей перестановки, но уже с порядком 3. \\
\\
\textbf{Задача 3.} Найти все перестановки, коммутирующие с $s = \{24315\} = (124)(3)(5)$
\begin{lstlisting}
[[(1, 1), (2, 2), (4, 4)], [(1, 2), (2, 4), (4, 1)], [(1, 3), (2, 3), (4, 3)],
[(1, 4), (2, 1), (4, 2)], [(1, 5), (2, 5), (4, 5)], [(3, 1), (3, 2), (3, 4)], 
[(3, 3)], [(3, 5)], [(5, 1), (5, 2), (5, 4)], [(5, 3)], [(5, 5)] ]
{3: 7, 1: 4}
\end{lstlisting}
Перестановки формируются из комбинаций циклов (остальные не подходят): $11 \rightarrow 22 \rightarrow 33, 12 \rightarrow 24 \rightarrow 41, 14 \rightarrow 21 \rightarrow 42, 33 + 55, 53 + 35$.
\[\{12345\}, \{12543\} = (35), \{24315\} = (124), \{24513\} = (124)(35) \]
\[\{41325\} = (142), \{41523\} = (142)(35) \]
\begin{center}
    \textsc{3. Упражнения}
\end{center}
(1) Какой цикловой структурой обладает перестановка в $S_7$, которая имеет максимальное число корней? Сколько их? \\ 
(2) Приведите пример перестановки из $S_7$, которая имеет максимальное число коммутирующих (помимо, разумеется, единичной)? Доказать и найти сколько коммутирующих перестановок к ней можно подобрать. \\
(3) Доказать, что из перестановки, которая имеет только один цикл четной длины, невозможно извлечь квадратный корень. \\
(4) Найти количество перестановок из $S_6$, у которых множество всевозможных степеней совпадает с множеством коммутирующих. Пример такой перестановки можно найти в задаче 2. Сформулировать критерий отбора подобных перестановок. Сработает ли он для группы $S_9$?
\\
\\
\textit{Ответы.} \\
(1) {\tiny 3-цикл, 10} \\
(2) {\tiny 2-цикл, 240} \\
(4) {\tiny 384}

\begin{center}
    \textsc{4. Сопряжённые перестановки}
\end{center}
Перестановки $\varphi$ и $\sigma$ называются \textit{сопряженными}, если существует такое $\tau$, что:
\[\varphi = \tau \sigma \tau^{-1}\]
Существует теорема о том, что перестановки $\varphi$ и $\sigma$ сопряжены в $S_n$ тогда и только тогда, когда они имеют одинаковую цикловую структуру. Покажем, как это выглядит на практике. Пусть $\sigma = (124)(56) \in S_6$.
\[\tau \sigma \tau^{-1} = (\tau(1)\tau(2)\tau(4))(\tau(5)\tau(6)) = \varphi \]
(по умолчанию $\tau(3) = 3$). \\
\textbf{Задача 4}. Найти количество всех возможных перестановок $\tau$, которые удовлетворяют выражению, $\varphi = \tau \sigma \tau^{-1}$  если $\varphi = (12)(34)(576)(8)(9)$ и $\sigma = (127)(34)(5)(6)(89)$. \\
\\
Существует достаточно элегантный способ, как быстро найти перестановку $\tau$. Пусть $\varphi = (12)(34)(576)(8)(9)$, а $\sigma = (127)(34)(5)(6)(89)$. Они однозначно сопряжены, так как имеют одинаковые цикловые структуры. Формируем таблицу, в \textit{первой строчке} которой запишем цикловую структуру именно $\varphi$! Во вторую строчку записываем циклы $\sigma$ так, чтобы 2-циклы (3-циклы, 1-циклы) стояли друг под другом. Тогда получается готовая запись $\tau$ в виде подстановки:
\[\begin{pmatrix}
1 & 2 & 3 & 4 & 5 & 7 & 6 & 8 & 9 \\
3 & 4 & 8 & 9 & 1 & 2 & 7 & 5 & 6 
\end{pmatrix} \Rightarrow \tau = \{348917256\} \]
Теперь нетрудно понять, сколько вообще возможно составить $\tau$. Пусть первая строчка определена. Выбрать место для первого 2-цикла: 2 способа, 2 способа записать его через циклический сдвиг, не нарушая логику (иными словами, (14) = (41), (127) = (712) = (271), это одинаковые циклы, но с точки зрения формирования $\tau$ разные). Существует 2 способа записать второй 2-цикл (для него место уже определено). Также и для 3-цикла, но существует 3 способа его записать. Два способа выбрать место для первого 1-цикла, оставшийся определен. Итого:
\[ 2 \times 2 \times 2 \times 3 \times 2 = 48 \] 
Согласно теореме мы можем определить, сколько элементов будет в множестве того или иного сопряженного класса. Теорема позволяет свести эту задачу к подсчету перестановок определенной цикловой структуры в $S_n$. Введем искусственное понятие \textit{паспорта перестановки.} Это структура вида $\{x_1: k_1, x_2: k_2, x_3: k_3, \dots \}$, где $x_i$ - цикл длины $i$, а $k_i$ - количество циклов длины $i$. В разделе будет предоставлено полное объяснение данной формулы.
\[ \boxed{\varepsilon(s) =  n! \cdot \prod_{i=1} \frac{1}{x_i^{k_i}\cdot k_i!}} \]

\newpage
\begin{center}
    \textsc{5. Подсчёт коммутирующих перестановок}
\end{center}
Теорему о сопряженных перестановках можно легко свести к частному случаю коммутирующих, когда $\varphi = \sigma$. Для $\sigma = (124)(56) \in S_6$. 
\[\tau \sigma \tau^{-1} = (\tau(1)\tau(2)\tau(4))(\tau(5)\tau(6)) = (124)(56) \]
Задача свелась к перебору частных случаев. 
\[ \tau(1) = 1, \tau(2) = 2, \tau(4) = 4; \ \tau(1) = 4, \tau(2) = 1, \tau(4) = 2; \]
\[ \tau(1) = 2, \tau(2) = 4, \tau(4) = 1 \]
Подчеркнем, что мы выбираем один из трех образов для 1, остальные элементы должны подстроиться автоматически, чтобы циклы были равны. C вторым циклом два возможных случая:
\[ \tau(5) = 5, \tau(6) = 6, \ \tau(5) = 6, \tau(6) = 5\]
Всего 6 вариантов, которые могут получиться из комбинаций циклов. Выведем \textit{общую формулу для вычисления количества коммутирующих перестановок.} 
Для перестановки $s$ вида \{$x_1$: $k_1$, $x_2$: $k_2$, $x_3$: $k_3$, \dots \} из группы $S_n$ количество коммутирующих перестановок равно:
\[\Phi(s) = k_1! \cdot k_2! \cdot k_3! \dots x_1^{k_1}\cdot x_2^{k_2} \cdot x_3^{k_3} \dots  (n - k_1x_1 - k_2x_2 - k_3x_3 - \dots)!\]
Расшифровка: $x_i$ - цикл длины $i$, где $i > 1$. $k_i$ - количество циклов длины $i$. \\
\textit{Пример}. Найти $\Phi(s)$, если $s \in S_n$ и имеет структуру: а) (3)-цикл?, $n = 7$; b) (2,2,2)-цикл? $n = 7$; c) (2, 2, 3)-цикл, $n = 9$. 
\[a) \ \Phi(s_a) = 1!\cdot 3 \cdot (7-3)! = 72\]
\[b) \ \Phi(s_b) = 3! \cdot 2 \cdot 2 \cdot 2 \cdot (7-6)! = 48\]
\[c) \ \Phi(s_c) = 2! \cdot 1!\cdot 2 \cdot 2 \cdot 3 \cdot (9-2\cdot 2 - 3)! = 48.\] 
Поймем, как работает формула. Вспоминаем, как в задаче 4 составлялась перестановка $\tau$: 2-циклы под 2-циклами, 3-цикл под 3-циклами и т.д. Однако циклы можно переставлять относительно друг друга, число место для расстановки и равняется числу циклов определенной длины. По такой же логике формируется и последний множитель, но только для 1-циклов. \\
Второй множитель это выбор того как будет строиться наша биекция. Выше мы выделили тот факт, что в цикле длины $n$ $n$ способами мы выбираем куда переходит один из участников цикла -- остальные уже определены. \\
Формула приведена в несколько избыточном виде, чтобы не забыть про 1-циклы, которые обычно опускают в записи перестановки. Аккуратная запись выглядит следующим образом:
\[s \in S_n \{x_1: k_1, x_2: k_2, x_3: k_3, \dots \} \Rightarrow \boxed{\Phi(s) = \prod_{i=1} x_i^{k_i} \cdot k_i!}\]
\textit{Проверьте упражнения 2 и 4.} \\ 
Осталось заметить, что $\Phi(s) \cdot \varepsilon(s) = n! = |S_n|$. 

\begin{center}
    \textsc{6. Подсчёт перестановок с определенной цикловой структурой}
\end{center}
Рассмотрим на конкретной задаче - поиск числа перестановок с паспортом {1: 1, 2: 2, 3: 2}. В обычной нотации (2, 2, 3, 3)-циклы в $S_{11}$. \\
\textbf{Этап 1.} Последовательный выбор чисел, которые войдут в первый 2-цикл, во второй 2-цикл и т.д. \\
\[ C^2_{11} \cdot C_9^2 \cdot C_7^3 \cdot C_4^3 = \frac{11!}{2! 2! \cdot 3! 3! \cdot 1!} = 277200  \]
Отмечаем, что множители вида $n-x$ в сочетании $C^x_n$ последовательно сокращаются. Нетрудно показать, что множитель, показывающий выбор чисел, запишется, как: (последним слагаемым вида $n-x$ и окажется число 1-циклов, что подчеркивается отдельной дробью).  
\[ \frac{n!}{(x_1!)^{k_1}(x_2!)^{k_2}(x_3!)^{k_3}\dots} \cdot \frac{1}{k_1!}\]
\textbf{Этап 2}. Закрутки циклов. \\
Снова для наглядности обратимся к примеру. (123) = (312) $\neq (132)$. Компоненты цикла мы уже выбрали, но нужно учесть расстановку элементов внутри него. Мы берем наименьшений элемент и ставим его на первое место. Остальные $n-1$ элементов переставляются как угодно. Эти циклы будут различны, так как их невозможно получить сдвигом. Для нашего примера: \\
\[ (3-1)!(3-1)!(2-1)!(2-1)! = 4 \]
В общем виде
\[\prod_{i=1}(x_i-1)!^{k_i}\]
\textbf{Этап 3.} Убираем повторяющиеся компоненты. \\
Рассматриваем циклы > 1, так как только они участвуют в записи. Число всевозможных перестановок из этих циклов - 4! (4 - сумма всех $k$: $x > 1$). Так как у нас есть циклы одинаковой длины, то некоторая часть перестановок была учтена больше чем один раз. Уникальные композиции расчитываются по формуле перестановок с повторениями. У нас 2 2-цикла и 2 3-цикла. Тогда:
\[\frac{4!}{2!2!} \]
Отношение уникальных к общим и показывает сколько настоящих перестановок есть среди тех, которые мы уже насчитали. В общем виде запишется как:
\[\frac{1}{k_2!k_3!\dots} \]
Тогда в нашей задаче: 
\[ \frac{1}{2! 2!} = \frac{1}{4}\]
Перемножаем все, что получили на 3 этапах, и получим : 277200. \\
В общем виде:
\[ \varepsilon(s) = \frac{n!}{\prod_i (x_i)!^{k_i}} \cdot \prod_i \frac{1}{k_i!} \cdot \prod_{i=1}(x_i-1)!^{k_i} = n! \cdot \prod_{i=1} \frac{1}{x_i^{k_i}\cdot k_i!} \]
\newpage
\begin{center}
    \textsc{7. Визуализация через графы}
\end{center}
\textbf{Сопряженная карта }- вершины соединены ребром, если объекты из этих вершин сопряжены. \\
\textbf{Коммутирующая карта} - вершины соединены ребром, если объекты, принадлежащие этим вершинам(циклам), коммутируют между собой. \\
\textbf{Корневая карта }- ориентированный граф. Если из вершины А исходит направленное ребро в вершину В, то существует такой представитель вершины В, что его степень (равная \textit{порядку корневой карты}) равен объекту вершины А. Мы рассматриваем карты второго порядка, оставаясь в рамках задачи о извлечении квадратного корня, однако ради интереса приведен пример карты третьего порядка для $S_6$. (фото 8). \\
\textbf{Композиционно-стабилизирующая карта} - вершины соединены ребрами, если существует такой представитель, что при умножении не меняет его цикловую структуру (стабилизирует). \\
\\
(1) Сколько перестановок из группы $S_8$ стабилизируют (т.е. оставляют неизменной цикловую структуру) (2,2)-цикл? \\
(2) Как можно охарактеризизовать вершины, соединенные ребром с нейтральным элементом на корневой карте порядка $n$? Сколько корней четвертой степени имеет единичная перестановка в $S_7$? \\

\end{document}